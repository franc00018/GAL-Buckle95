\documentclass{report}
\usepackage{Sweave}
\usepackage{graphicx}
\usepackage[francais]{babel} 
\usepackage[utf8]{inputenc}
\usepackage[T1]{fontenc} 
\usepackage{amsmath} 
\usepackage{amsfonts}
\usepackage{verbatim} 
\usepackage{float} 
\usepackage{hyperref}
\usepackage{scrtime}

\begin{document}
\input{GAL-Buckle95-concordance}

\title{GAL Buckle 95}
\author{François Pelletier}
\maketitle
\tableofcontents

\section{Chargement des paquets}
\begin{Schunk}
\begin{Sinput}
> setwd("~/git/GAL-Buckle95/")
> library(actuar)
> library(MASS)
> library(xtable)
> library(multicore)
> library(moments)
> library(TTR)
> library(FourierStuff)
> library(GeneralizedAsymmetricLaplace)
> library(GMMStuff)
> library(OptionPricingStuff)
> library(QuadraticEstimatingEquations)
\end{Sinput}
\end{Schunk}

\section{Constantes et données}

\begin{Schunk}
\begin{Sinput}
> #Nombre de décimales affichées
> options(digits=6)
> #Marge pour intervalles de confiance
> alpha.confint <- 0.05 
> #Marge pour test d'hypothèses
> alpha.test <- 0.05
> #Chargement des données
> RETURNS <- head(read.csv("abbeyn.csv",sep="\t",header=TRUE)[,1],-1)
\end{Sinput}
\end{Schunk}

\section{Test de normalité}

\begin{Schunk}
\begin{Sinput}
> EppsPulley.test(RETURNS)
\end{Sinput}
\begin{Soutput}
Epps-Pulley Normality test

 T: 0.626033
 T*: 0.635568
p-value: 0.007178

$Tstat
[1] 0.626033

$Tmod
[1] 0.635568

$Zscore
[1] 2.44824

$Pvalue
[1] 0.00717788

$Reject
[1] TRUE
\end{Soutput}
\end{Schunk}

\section{Données mises à l'échelle}

\begin{Schunk}
\begin{Sinput}
> scaledRETURNS <- as.vector(scale(RETURNS))
\end{Sinput}
\end{Schunk}

\section{Estimation}

\begin{Schunk}
\begin{Sinput}
> ## Point de départ	
> pt.depart <- startparamGAL(scaledRETURNS)
> ## Fonctions pour les moments
> meanQEE <- function(param) mGAL(param,1)
> varianceQEE <- function(param) cmGAL(param,2)
> sdGEE <- function(param) sqrt(cmGAL(param,2))
> skewnessGEE <- function(param) cmGAL(param,3)
> kurtosisGEE <- function(param) cmGAL(param,4)
> ## Fonctions pour les dérivées
> dmeanQEE <- function(param) dmGAL(param,1)
> dsdGEE <- function(param) dmGAL(param,2)
> ## Estimation gaussienne
> optim1 <- optim(pt.depart,obj.gauss,scaledRETURNS,meanQEE,varianceQEE,dmeanQEE,dsdGEE)